\documentclass{article}
\usepackage[utf8]{inputenc}
\usepackage{geometry}
 \geometry{
 a4paper,
 total={170mm,257mm},
 left=20mm,
 top=20mm,
 }

\title{3GPP LTE : Cramer Rao Lower Bound on TOA}
\author{Siddhant}
\date{April 2020}

\begin{document}

\maketitle
\section{CRLB for OFDM Based Systems}
\subsection{Derivation: Fisher Information}
$s(t)$ is the OFDM wave being transmitted by Reference Cell. $\tau$ is the time delay and channel is assumed to be $1$. Received Samples can be given by the following :$$y[n] = s(nT_s - \tau) + z[n],$$where z[n] are iid AWGN noise samples with mean $0$ and variance $\sigma^2$.  
Joint Conditional PDF of $\mathbf y|\tau$ can be given by the following:
$$p(\mathbf y|\tau) = \frac{1}{(\pi \sigma^2)^N} .exp(-\frac{1}{\sigma^2}\sum^{N}_{n=1}(s(n T_s - \tau) - y[n])^2)$$

And log likelihood will become :
$$ln(p(\mathbf y|\tau)) = -N ln(\pi \sigma^2) -\frac{1}{\sigma^2}\sum^{N}_{n=1}(s(nT_s - \tau) - y[n])^2$$

Fisher Information $\mathbf J(\tau)$ will then be given by:
$$\mathbf{J}(\tau) = \mathbf E\{ - \mathbf{\frac{d^2}{d^2\tau}}ln((p(\mathbf y|\tau)) \}$$

Simplify the above to get :
$$\mathbf{J}(\tau) = \frac{2}{sigma^2}\sum^{N}_{n=1}\mathbf{\frac{d}{d\tau}}s(n T_s - \tau).\mathbf{\frac{d}{d\tau}}s^*(n T_s - \tau)$$
Since, $s(t) = \frac{1}{\sqrt{N}}\sum^{\frac{N}{2} - 1}_{k=-\frac{N}{2}}S[k]e^{j2\pi k\Delta F t}$, where $\Delta F$ is subcarrier spacing of OFDM system. Hence, $s(n T_s - \tau) = \frac{1}{\sqrt{N}}\sum^{\frac{N}{2} - 1}_{k=-\frac{N}{2}}S[k]e^{j2\pi k\Delta F (n T_s - \tau)}$. Substitute this $s(n T_s - \tau)$ in the above equation to get : 
$$\mathbf{J}(\tau) = \frac{2}{N.\sigma^2} \sum^{N}_{n=1}\{\frac{d}{d\tau}\sum^{\frac{N}{2} - 1}_{k=-\frac{N}{2}}S[k]e^{j2\pi k\Delta F (n T_s - \tau)}\}.\{\frac{d}{d\tau}\sum^{\frac{N}{2} - 1}_{l=-\frac{N}{2}}S[k]e^{j2\pi l\Delta F (n T_s - \tau)}\}$$

After carrying out differentiation and taking all terms which are not running in the sum out of the summation, we get the following :
$$\mathbf{J}(\tau) = \frac{2}{N\sigma^2}(-j2\pi \Delta F).(j2\pi \Delta F)\sum^{N}_{n=1} \{\sum^{\frac{N}{2} - 1}_{k=-\frac{N}{2}}k S[k]e^{j2\pi \Delta F k n T_s}.\sum^{\frac{N}{2} - 1}_{l=-\frac{N}{2}}l S*[l]e^{-j2\pi \Delta F l n T_s}\}$$

Use orthogonality of Fourier Basis and assume $S[k] = 1$, to get the following:
$$\mathbf{J}(\tau) = \frac{8\pi^2 \Delta F^2}{N\sigma^2} \sum^{N}_{n=1} \sum^{\frac{N}{2} - 1}_{k=-\frac{N}{2}}N k^2 = \frac{8\pi^2 \Delta F^2}{\sigma^2} \sum^{N}_{n=1} \sum^{\frac{N}{2} - 1}_{k=-\frac{N}{2}}k^2 = \frac{8\pi^2 \Delta F^2 N}{\sigma^2}\sum^{\frac{N}{2} - 1}_{k=-\frac{N}{2}}k^2$$

Computing $\sum^{\frac{N}{2} - 1}_{k=-\frac{N}{2}}k^2$ by substituting $m=k+\frac{N}{2}$, gives the following :

$$\sum^{N-1}_{m=0}(m-\frac{N}{2})^2 = \frac{N^3}{12} + \frac{N}{6} \approx \frac{N^3}{12}$$

Finally, plug this approximated sum into Fisher Information sum to get the following :
$$\mathbf{J}(\tau) = \frac{8\pi^2 \Delta F^2 N}{\sigma^2}\sum^{\frac{N}{2} - 1}_{k=-\frac{N}{2}}k^2 \approx \frac{8\pi^2 \Delta F^2 N}{\sigma^2} \frac{N^3}{12},$$
We know that Bandwidth of OFDM is $\Delta F.N$, hence $(\Delta F . N)^2$ can be considered as Squared Bandwidth. Also We have considered $|S[k]|^2 = 1$, which will mean Pre-Correlation Signal Power is 1. If instead Pre-Correlation Signal Power was a constant $P$, then Post-Correlation signal power would have been $N.P$. And then the Fisher Information expression becomes :
$$\mathbf{J}(\tau) = \frac{2\pi^2 (\Delta F.N)^2}{3}. \frac{N^2 P}{\sigma^2} = N. \frac{2\pi^2}{3} (\Delta F.N)^2 SNR_{corr}$$

\subsection{CRLB : Expression}
As, CRLB on uncertainty of the estimated parameter $\tau$ is lower bounded by reciprocal of Fisher Information, CRLB can be given by following:
$$var(\hat{\tau}) \ge \frac{1}{\mathbf J(\tau)},$$
$$or, var(\hat{\tau}) \ge \frac{3}{2N\pi^2}\frac{1}{(\Delta F.N)^2}.\frac{1}{SNR_{corr}}$$

\section{PRS Based New CRLB Calculation}
\subsection{Introduction : OFDM Waveform Structure}
Assume $x[n] = \sqrt{\frac{2C}{N_c}}\sum_{k \in N_a}p_k.d_k.e^{j2\pi n k/N_c}$, where $C$ is the power in the bandpass signal, $p_k^2$ is relative power weight of $k^{th}$ tone, $d_k$ is Gold-sequence Symbol on $k^{th}$ tone and $N_c$ is number of sub-carriers. $p_k^2$ is constrained by $\sum_{k \in N_a}p_k^2 = N_c$, so that the nominal signal power is $C$.

\subsection{CRLB Analysis : General Range Estimation in AWGN}
Now if we go by the analysis of Range estimation given in 'Fundamentals of Statistical Signal processing: Estimation Theory',Steven M Kay, then, CRLB for estimator $\hat{\tau}$ will be given by the following:
$$var(\hat{\tau}) \ge CRLB(\tau) = \frac{1}{\frac{E_s}{N_o/2}\cdot \bar{F}^2},$$
Here, $E_s = C\cdot T_s$, is signal Energy and $\frac{N_o}{2}$ is Noise Energy, which makes $E_s/(N_o/2)$ as SNR, and $\bar{F}^2$ is Gabor's Bandwidth or Mean Squared bandwidth. This mean Squared Bandwidth for OFDM is defined as the following:
$$\bar{F}^2 = \frac{\int^{\infty}_{-\infty}(2\pi f)^2 \left|X(f)\right|^2\cdot df}{\int^{\infty}_{-\infty}\left|X(f)\right|^2\cdot df}$$

\subsection{CRLB Derivation : Pilot Based OFDM System}

\end{document}
